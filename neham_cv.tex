%-------------------------
% Resume in Latex
%------------------------

\documentclass[letterpaper,10pt]{article}

\usepackage{latexsym}
\usepackage[empty]{fullpage}
\usepackage{titlesec}
\usepackage{marvosym}
\usepackage[usenames,dvipsnames]{xcolor}
\usepackage{verbatim}
\usepackage{enumitem}
\usepackage[pdftex, hidelinks]{hyperref}
\hypersetup{
    colorlinks=true,
    linkcolor=blue,
    citecolor=blue,
    urlcolor=blue
}

\usepackage{fancyhdr}
\usepackage{fontawesome}
\usepackage[charter]{mathdesign} % Bitstream Charter
% \usepackage{newpxtext,newpxmath} % Palatino
\pagestyle{fancy}
\fancyhf{} % clear all header and footer fields
\fancyfoot{}
\renewcommand{\headrulewidth}{0pt}
\renewcommand{\footrulewidth}{0pt}

% Adjust margins
\addtolength{\oddsidemargin}{-0.50in}
\addtolength{\evensidemargin}{-0.50in}
\addtolength{\textwidth}{1in}
\addtolength{\topmargin}{-.5in}
\addtolength{\textheight}{1.0in}

\urlstyle{same}

\raggedbottom
\raggedright
\setlength{\tabcolsep}{0in}

% Sections formatting
\titleformat{\section}{
  \color{teal}\vspace{-6pt}\scshape\raggedright\large
}{}{0em}{}[\color{Gray}{\titlerule[2pt]}\vspace{-5pt}]

%-------------------------
% Custom commands
\newcommand{\resumeItem}[2]{
  \item\small{
    \textbf{#1}{: #2 \vspace{-2pt}}
  }
}

\newcommand{\resumeItemNoBullet}[2]{
  \item[]\small{
    \hspace{-9pt}\textbf{#1}{: #2 \vspace{-6pt}}
  }
}

\newcommand{\resumeSubheading}[4]{
  \vspace{-1pt}\item[]
  \begin{tabular*}{0.98\textwidth}{l@{\extracolsep{\fill}}r}
      \hspace{-10pt}\textbf{#1} & #2 \\
      \hspace{-10pt}\textit{\small#3} & \textit{\small #4} \\
    \end{tabular*}\vspace{-5pt}
}

\newcommand{\resumeSubheadingCustom}[6]{
  \vspace{-1pt}\item[]
  \begin{tabular*}{0.98\textwidth}{l@{\extracolsep{\fill}}r}
      \hspace{-10pt}\textbf{#1} & #2 \\
      \hspace{-10pt}\textit{\small#3} & \textit{\small #4} \\
      \hspace{-10pt}\small#5 & \textit{\small #6} \\
    \end{tabular*}\vspace{-5pt}
}
\newcommand{\resumeSubheadingNew}[4]{
  \vspace{-1pt}\item[]
  \begin{tabular*}{0.98\textwidth}{l@{\extracolsep{\fill}}r}
      \hspace{-10pt}\textbf{#1} | \textit{#3} & #2 \\
      ~ & \textit{\small #4} \\
    \end{tabular*}\vspace{-20pt}
}


\newcommand{\resumeSubItem}[2]{\resumeItem{#1}{#2}\vspace{-4pt}}

\renewcommand{\labelitemii}{$\circ$}

\newcommand{\resumeSubHeadingListStart}{\begin{itemize}[leftmargin=*]}
\newcommand{\resumeSubHeadingListEnd}{\end{itemize}}
\newcommand{\resumeItemListStart}{\begin{itemize}}
\newcommand{\resumeItemListEnd}{\end{itemize}\vspace{-5pt}}


% custom commands
\newcommand{\shorterSection}[1]{\vspace{-10pt}\section{#1}}

%-------------------------------------------
%%%%%%  CV STARTS HERE  %%%%%%%%%%%%%%%%%%%%%%%%%%%%

\begin{document}

%----------HEADING-----------------
\begin{center}
  \small \textbf{\huge NEHAM JAIN} \\ 
  \vspace{0.2cm}
  \faEnvelopeO ~  \href{mailto:nehamjain2002@gmail.com}{\color{blue}\underline{nehamjain2002@gmail.com}}~
   \faGithub~ \href{https://github.com/nehamjain10}{\color{blue}\underline{nehamjain10}}  ~
  \faLinkedin~
  \href{https://www.linkedin.com/in/neham-jain/}{\color{blue}\underline{Neham Jain}}
  \faPhone~ {\color{blue}\underline{+1-901-208-2738}}
\end{center}
\vspace{.15cm}
%-----------EDUCATION-----------------
\shorterSection{Education}
  \resumeSubHeadingListStart
      \resumeSubheadingCustom
      {Carnegie Mellon University}{May 2025}
      {Master of Science in Robotics, \textbf{Advisor: Prof. Ioannis Gkioulekas};  \textbf{GPA: 4.11/4.0}}{}
      {\textbf{Relevant Coursework}: Deep Learning Systems, Learning for 3D Vision, Robot Learning, Physics based Rendering}{}
      
      
    \resumeSubheadingCustom
      {Indian Institute of Technology (IIT), Madras }{June 2023}
      {Bachelor of Technology in Electrical Engineering;  \textbf{CGPA: 9.24/10}}{\textbf{Minor:} Machine Learning}{\textbf{Relevant Coursework}: Modern Computer Vision, Machine Learning, Computational Photography}{}

  \resumeSubHeadingListEnd

%-----------EXPERIENCE-----------------

\shorterSection{Skills}
     \textbf{Languages}{: Python, C++, SQL, CUDA C, Java}
     \hfill
     \textbf{Libraries}{: PyTorch, JAX, ROS, OpenGL, Vulkan, NumPy}
    
\vspace{2mm}
\shorterSection{Professional Experience}
  \resumeSubHeadingListStart

   \resumeSubheading
   {Meta Reality Labs | Data Pipelines for Training Embodied AI Agents}{June 2025 - Present}{Research Engineer}{Pittsburgh, PA}
\begin{itemize}[leftmargin=*]
    \item \textbf{\href{https://arxiv.org/pdf/2510.16258}{Embody3D} dataset release}: built QA tools, coordinated annotation of \textbf{1.6M videos}, and curated \textbf{500 hours} of high-quality data—\textbf{187 GitHub stars}, \textbf{3 citations}, adopted by internal teams for training.
    \item Built \textbf{automated quality filtering} algorithms to eliminate annotation bottlenecks and ensure high-quality data for training.
    \item Achieved \textbf{75\% reduction} in tracking error rate by designing a \textbf{self-supervised data generation} to train keypoint model.
\end{itemize}

   \resumeSubheading
   {Adobe Research |  Scalable Pipeline for Training 3D Foundation Models}{Summer 2024}{Research Engineer Intern}{San Francisco, CA}
\begin{itemize}[leftmargin=*]
    \item Created a \textbf{pipeline} to convert \textbf{synthetic mesh assets} into \textbf{relightable Gaussian Splats} for training \textbf{3D Foundation Models}.
    \item Wrote \textbf{custom CUDA kernels} to replace \textbf{spherical harmonics} with \textbf{neural features} for \textbf{relightability} and \textbf{faster training}.
    \item Scaled the training \textbf{pipeline} to 32 nodes and \textbf{improved training time} from \textbf{6 minutes} per asset to \textbf{30 seconds}. 
\end{itemize}

   \resumeSubheading
   {Adobe Research | Counterfactual Explanations of Visual Recommender Systems \href{https://dl.acm.org/doi/10.1145/3589335.3651484}{[Paper]}
}{Summer 2022}{Research Scientist Intern}{Bangalore, India}
\begin{itemize}[leftmargin=*]
    \item Worked on \textbf{counterfactual explanations} for \textbf{recommendation systems} based on \textbf{vision language models}.
    \item Computed the \textbf{minimal meaningful perturbation} to an item’s \textbf{image-embedding} that would remove it from a user’s recommended list. Used \textbf{CLIP} to connect \textbf{image features} to \textbf{textual labels} to lend meaning to the \textbf{perturbations}.
    \item Work was accepted at \textbf{WWW 2024} and a \textbf{patent} has been filed for our work at \textbf{USPTO}.
\end{itemize}


   \resumeSubheading
   {Subex AI Labs $|$ Information extraction from unstructured invoices}{Summer 2021}{Machine Learning Intern}{Bangalore, India}
\begin{itemize}[leftmargin=*]
    \item Developed an end to end pipeline for \textbf{multilingual invoice understanding} and \textbf{table extraction} using object detection.
    \item Fine-tuned a \textbf{multimodal document transformer} model (\textbf{Layout-LMv2}) to extract form fields using \textbf{token classification}.


\end{itemize}

\resumeSubHeadingListEnd

\shorterSection{Relevant Projects}
  \resumeSubHeadingListStart


   \resumeSubheading
   {Masters Thesis $|$  3D Gaussian splatting for reconstruction of wildfire scenes \href{https://imaging.cs.cmu.edu/smokeseer/}{[3DV 2026 Poster]}}{Fall 2023 - Present}{Guide: Dr. Ioannis Gkioulekas}{Pittsburgh, PA}
\begin{itemize}[leftmargin=*]
    \item Decomposed \textbf{wildfire scenes} into \textbf{smoke and surface Gaussians} for \textbf{smoke-free rendering} to assist firefighting.
    \item Integrated \textbf{fluid particle hydrodynamics} into the \textbf{3D Gaussian Splatting} pipeline to accurately model the \textbf{temporal dependence of smoke}, resulting in more \textbf{realistic} and \textbf{artifact-free} reconstructions.
    \item Adapted \textbf{Mast3r-SfM} to perform \textbf{localization} of the  \textbf{RGB} and \textbf{thermal} cameras in the same coordinate frame. 

\end{itemize}


%    \resumeSubheading
%    {RGB-Thermal Sensor Fusion for ADAS Applications}{August 2022 - May 2023}{Guide: Prof. Aswin Sankaranarayanan (Carnegie Mellon University), Prof. Kaushik Mitra (IIT Madras)}{Pittsburgh, PA}
% \begin{itemize}[leftmargin=*]
%     % modify this to add name of project
%     \item Proposed a \textbf{physics based approach} to create a \textbf{low-light RGB, well-lit RGB} and \textbf{thermal image} of the same scene.
%     \item Implemented a novel \textbf{cross-attention architecture} to \textbf{super-resolve} thermal images using RGB as a \textbf{guide}.
%     \item Used .
    
    
% \end{itemize}

 \resumeSubheading
   {Needle - A Deep Learning Framework in Python \href{https://colab.research.google.com/drive/13xndhmKBW1cjgtueA5-biCThGCdB6waA?usp=sharing}{[Report]}}{Fall 2024}{Course Project for Deep Learning Systems }{Pittsburgh, PA}
\begin{itemize}[leftmargin=*]
    \item  Created a general purpose \textbf{library} for \textbf{differentiable programming} from scratch with support for \textbf{C} and \textbf{CUDA} backends 
    \item Implemented features such as \textbf{gradient accumulation} and \textbf{distributed training (NCCL)} for training larger models.
        \item Trained \textbf{Llama-3} architecture across \textbf{8 4090s} nodes on \textbf{OpenWebText} using custom \textbf{NCCL} and \textbf{Gloo} backend.

\end{itemize}

% \resumeSubheading
% {Physically Based Ray Tracer with Spectral Rendering}{Spring 2024}{Course Project for Physics Based Rendering}{Pittsburgh, PA}
% \begin{itemize}[leftmargin=*]
% \item Implemented a \textbf{path tracer} with \textbf{multiple importance sampling} achieving realistic global illumination.

% \item Integrated \textbf{spectral rendering} using measured wavelength data to simulate accurate light dispersion and chromatic effects.
% \end{itemize}


  \resumeSubheading
   {Multiview Diffusion models for image to image translation}{Spring 2024}{Course Project for Learning for 3D Vision}{Pittsburgh, PA}
\begin{itemize}[leftmargin=*]
    \item Developed a \textbf{multiview-aware diffusion model} for translating between domains, such as day-to-night.
    \item Designed a \textbf{conditional U-Net architecture} with multiview feature aggregation using \textbf{epipolar constraints}.
    \end{itemize}

%    \resumeSubheading
%    {Geometry Constrained Gaussian Splatting \href{https://gcgsplatting.github.io/}{[Report]}}{Fall 2023}{Course Project for Geometry-based Methods in Vision }{Pittsburgh, PA}
% \begin{itemize}[leftmargin=*]
%     \item Incorporated geometric constraints such as \textbf{epipolar loss} and \textbf{depth loss} to improve \textbf{few-shot gaussian splatting}.
%     \item Achieved a performance boost of about \textbf{5\%} across \textbf{PSNR, SSIM, LPIPS} on the NeRF LLFF dataset.

% \end{itemize}



%    \resumeSubheading
%    {Undergraduate Research Project $|$ Color Restoration for Underwater Images \href{https://arxiv.org/abs/2211.14821}{[Paper]}}{Fall 2021 - Fall 2022}{Guide: Prof. Kaushik Mitra}{Chennai, India}
% \begin{itemize}[leftmargin=*]
%     \item Created a \textbf{multimodal domain adaptation (GANs)} technique to create a dataset for \textbf{different water conditions}.
%     \item Existing \textbf{algorithms} trained on our \textbf{dataset} attained superior performance on underwater \textbf{image restoration} task.
% \end{itemize}

% \resumeSubHeadingListEnd


% \shorterSection{Relevant Projects}
%   \resumeSubHeadingListStart

 


\resumeSubHeadingListEnd
% \shorterSection{Relevant Coursework}
%     \resumeSubHeadingListStart

%       \item{ \textbf{Artificial Intelligence/Machine Learning:} Fundamentals of Deep Learning | Reinforcement Learning | Introduction to Machine Learning |  | Nonlinear optimisation	 | Information Theory}
%       \item{ \textbf{Mathematics:} Mathematical Foundations of Robotics | Probability, Statistics and Stochastic Processes | Linear Algebra for Engineers | Applied Statistics | Functions of Several Variables | Series and Matrices  }
%       \item{ \textbf{Computer Vision/Graphics:} Modern Computer Vision  | Computational Photography | Physics-based Rendering | Geometry based Methods in Vision | Computer Graphics}
 
% \resumeSubHeadingListEnd


%-----------SKILLS-----------------


\end{document}


 